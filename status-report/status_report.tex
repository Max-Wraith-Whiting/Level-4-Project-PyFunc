    
\documentclass[11pt]{article}
\usepackage{times}
    \usepackage{fullpage}
    
    \title{Py-Func (Name T.B.A.) - A Python-Like Functional Programming Language}
    \author{Max Wraith-Whiting — 2560800W}

    \begin{document}
    \maketitle

\section{Status report}

\subsection{Proposal}\label{proposal}

\subsubsection{Motivation}\label{motivation}

    Python is an immensely popular and versatile programming language known for its readability and ease of use.
    However, it is not primarily designed for functional programming, which can lead to challenges when writing purely functional code.
    Python includes statements, mutable data structures, and lacks a robust type system.
    These limitations can hinder the development of large-scale, maintainable, and error-free functional codebases.
    This is in direct contrast to functional programming languages which have gained traction in recent years due to their many advantages, including enhanced code reliability, better parallelism, and easier testing.

    So, the fundamental idea behind this project is to leverage both Python's and Functional programming languages advantageous features into a single language.
    
\subsubsection{Aims}\label{aims}

    The project aims to produce a language that capitalises on the benefits of both Python's friendly syntax \emph{and} functional programming's many advantages.
    The language will need to provide features such as immutability, first-class functions, and a strong type system based on Hindley-Milner to make it suitable for functional development.
    By creating a Python-like functional programming language, we aim to provide the following benefits:

    \begin{itemize}
        \item \emph{Readability and Ease of Learning:} Leveraging Python's familiar syntax will lower the barrier to entry for developers new to functional programming, making it easier to transition to this paradigm.
        \item \emph{Error Prevention:} A robust type system with Hindley-Milner type inference will catch errors at compile-time, reducing runtime errors and enhancing code reliability.
        \item \emph{Improved Code Maintenance:} Functional programming techniques, like immutability and parametric polymorphism, simplify code maintenance and debugging.
    \end{itemize}

\subsection{Progress}\label{progress}

\begin{itemize}
    \item Implementation language chosen: Ocaml
    \item Basic structure of language pipeline draw out.
    \item Significant background research conducted on type-theory and the implementation of a Hindley-Milner type system.
    \item Created an intermediate representation (IR) specification. (Which is effectively an enhanced form of the lambda calculus, including binary-operations, if-else expressions and let expressions.)
    \item IR lexer and parser created to generate abstract syntax trees.
    \item Hindley-Milner type-checker implemented.
    \item Basic skeleton of the IR interpreter.
\end{itemize}

\subsection{Problems and risks}\label{problems-and-risks}

\subsubsection{Problems}\label{problems}


\begin{itemize}
    \item The topic of type-systems has a fog of difficult to interpret documentation and its own formal language to describe the function of these systems. A lot of time was needed to parse the information.
    \item Developing in Ocaml has been slower than anticipated due to multitudinous reasons. (Poor-error messages, difficulty managing dependencies, etc.)
    \item The existing type-checker does not recognise decimal numbers and the IR does not yet support them either.
\end{itemize}

\subsubsection{Risks}\label{risks}

\begin{itemize}
    \item Large pool of available syntax in Python to include. \textbf{Mitigation:} will have to restrict to only the most obvious syntax being flexible depending on work progress in Semester 2.
    \item Difficult to measure how \emph{"successful"} a programming language is. \textbf{Mitigation:} will measure against existing Python specification and code snippets to do in-situ comparisons.
    \item The type-checker is difficult to work with, and therefore the required future extensions will be an unknown amount of work. \textbf{Mitigation:} will block out additional time for these extensions than strictly necessary.
\end{itemize}

\subsection{Plan}\label{plan}

   \begin{itemize}
    \item Week 1-2: Complete Interpreter and extend Hindley-Milner type-checker to include float values.
    \begin{description}
        \item [— Deliverable:] A fully functional interpreter for the existing IR, including float or int values in a boxed type (number).
    \end{description}

    \item Week 3-4: Work on the Python-esque frontend to the language.
    \begin{description}
        \item[— Deliverable:] A functional lexer and parser, with space for expansion, that encompasses basic Python syntax.
    \end{description}

    \item Week 5-6: Implement the Python AST converter for the IR.
    \begin{description}
        \item[— Deliverable:] Complete converter for the existing syntax (at this point of the project).
    \end{description}

    \item Week 7-8: Further expansion of the frontend to include more Python features. Begin comparative testing and dissertation.
    \begin{description}
        \item[— Deliverable:] A more complete representation of Python in the new language. (This will likely not include more advanced features. E.g., Classes.)
    \end{description}

    \item Week 9-10: Finalisation of Project and Dissertation.
    \begin{description}
        \item[— Deliverable:] A complete project (within reason) and dissertation.
    \end{description}
   \end{itemize} 
    
    \end{document}
